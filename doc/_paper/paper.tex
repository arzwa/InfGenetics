%%
%% Copyright 2022 OXFORD UNIVERSITY PRESS
%%
%% This file is part of the 'oup-authoring-template Bundle'.
%% ---------------------------------------------
%%
%% It may be distributed under the conditions of the LaTeX Project Public
%% License, either version 1.2 of this license or (at your option) any
%% later version.  The latest version of this license is in
%%    http://www.latex-project.org/lppl.txt
%% and version 1.2 or later is part of all distributions of LaTeX
%% version 1999/12/01 or later.
%%
%% The list of all files belonging to the 'oup-authoring-template Bundle' is
%% given in the file `manifest.txt'.
%%
%% Template article for OXFORD UNIVERSITY PRESS's document class `oup-authoring-template'
%% with bibliographic references
%%

%%%CONTEMPORARY%%%
%\documentclass[,webpdf,contemporary,large]{oup-authoring-template}%
%\documentclass[unnumsec,webpdf,contemporary,large,namedate]{oup-authoring-template}% uncomment this line for author year citations and comment the above
%\documentclass[unnumsec,webpdf,contemporary,medium]{oup-authoring-template}
%\documentclass[unnumsec,webpdf,contemporary,small]{oup-authoring-template}

%%%MODERN%%%
%\documentclass[webpdf,modern,large,namedate]{oup-authoring-template}% uncomment this line for author year citations and comment the above
\documentclass[unnumsec,webpdf,modern,large]{_oup-authoring-template}
%\documentclass[unnumsec,webpdf,modern,medium]{oup-authoring-template}
%\documentclass[unnumsec,webpdf,modern,small]{oup-authoring-template}

%%%TRADITIONAL%%%
%\documentclass[unnumsec,webpdf,traditional,large]{oup-authoring-template}
%\documentclass[unnumsec,webpdf,traditional,large,namedate]{oup-authoring-template}% uncomment this line for author year citations and comment the above
%\documentclass[unnumsec,namedate,webpdf,traditional,medium]{oup-authoring-template}
%\documentclass[namedate,webpdf,traditional,small]{oup-authoring-template}

%\onecolumn % for one column layouts

%\usepackage{showframe}
\usepackage{cleveref}

\graphicspath{{Fig/}}

% line numbers
%\usepackage[mathlines, switch]{lineno}
%\usepackage[right]{lineno}

\theoremstyle{thmstyleone}%
\newtheorem{theorem}{Theorem}%  meant for continuous numbers
%%\newtheorem{theorem}{Theorem}[section]% meant for sectionwise numbers
%% optional argument [theorem] produces theorem numbering sequence instead of independent numbers for Proposition
\newtheorem{proposition}[theorem]{Proposition}%
%%\newtheorem{proposition}{Proposition}% to get separate numbers for theorem and proposition etc.
\theoremstyle{thmstyletwo}%
\newtheorem{example}{Example}%
\newtheorem{remark}{Remark}%
\theoremstyle{thmstylethree}%
\newtheorem{definition}{Definition}

\DeclareMathSymbol{\shortminus}{\mathbin}{AMSa}{"39}
\newcommand{\Ex}{\mathbb{E}}
\newcommand{\Var}{\mathrm{Var}}
\newcommand{\var}{\mathrm{Var}}
\newcommand{\HW}{\mathrm{HW}}
\newcommand{\RI}{\mathrm{RI}}
\newcommand{\Bin}{\text{Binomial}}
\newcommand{\Uniform}{\text{Uniform}}
\newcommand{\Normal}{\mathcal{N}}
\newcommand{\Beta}{\text{Beta}}
\newcommand{\Exp}{\text{Exponential}}
\newcommand{\Gam}{\text{Gamma}}
\newcommand{\Bfun}{\mathrm{B}}
\newcommand{\eps}{\epsilon}
\newcommand{\all}{A}
\newcommand{\erf}{\mathrm{erf}}
\newcommand{\erfi}{\mathrm{erfi}}
\newcommand{\fix}{\mathrm{fix}}
\newcommand{\logit}{\mathop{\mathrm{logit}}}
\usepackage{caption}
\captionsetup[figure]{labelfont=bf,font=small}
\captionsetup[ table]{labelfont=bf,font=small}
\captionsetup[longtable]{labelfont=bf,font=small}
\usepackage{float,soul}
\usepackage[normalem]{ulem}
\makeatletter
\def\fps@figure{tb}
\makeatother
\usepackage{algorithm}
\usepackage{algpseudocode}
\usepackage{lipsum}
\usepackage{tikz}


\begin{document}

\journaltitle{Journal Title Here}
\DOI{HERE}
\copyrightyear{2024}
\pubyear{}
\access{}
\appnotes{}

\firstpage{1}

%\subtitle{Subject Section}

\title[Infinitesimal model for polyploids]{
    The infinitesimal model for autopolyploids and mixed-ploidy populations}

\author[1,$\ast$]{Arthur Zwaenepoel}

\authormark{Zwaenepoel}

\address[1]{
    \orgdiv{CNRS}, 
    \orgname{Univ. Lille, UMR 8198 -- Evo-Eco-Paleo}, 
    \orgaddress{\postcode{F-59000}, \state{Lille}, \country{France}}}

\corresp[$\ast$]{
\href{email:arthur.zwaenepoel@univ-lille.fr}{email:arthur.zwaenepoel@univ-lille.fr, }}

\received{Date}{0}{Year}
\revised{Date}{0}{Year}
\accepted{Date}{0}{Year}

%\editor{Associate Editor: Name}

%\abstract{
%\textbf{Motivation:} .\\
%\textbf{Results:} .\\
%\textbf{Availability:} .\\
%\textbf{Contact:} \href{name@email.com}{name@email.com}\\
%\textbf{Supplementary information:} Supplementary data are available at \textit{Journal Name}
%online.}

\abstract{
    We define the infinitesimal model of quantitative genetics (\textit{sensu}
    \cite{barton2017}) for the inheritance of an additive quantitative trait in
    a mixed-ploidy population consisting of diploid, triploid and autotetraploid
    individuals producing haploid and diploid gametes.
    We implement efficient simulation methods and use these to study the
    quantitative genetics of mixed-ploidy populations and the establishment of
    tetraploids after an environmental challenge and in a new habitat.
}
\keywords{Polyploidy, quantitative genetics, infinitesimal model}

% \boxedtext{
% \begin{itemize}
% \item Key boxed text here.
% \item Key boxed text here.
% \item Key boxed text here.
% \end{itemize}}

\maketitle

\section*{Introduction}

\section*{Model \& Methods}

We first develop the infinitesimal model (in the sense of \cite{barton2017},
i.e.~the `Gaussian descendants' approximation \citep{turelli2017}) for an
autotetraploid population. We then consider mixed-ploidy populations in which
diploids, triploids and tetraploids coexist and interbreed through the
production of haploid and diploid gametes that combine freely.

{\subsection{The infinitesimal model}\label{the-infinitesimal-model}}

Consider a population which expresses a quantitative trait determined by a
large number of additive loci of small effect.
The infinitesimal model approximates the inheritance of such a trait by
assuming that the trait value \(Z_{ij}\) of a random offspring from parents
with trait values \(z_i\) and \(z_j\) follows a Gaussian distribution with mean
equal to the midparent value and variance which is independent of the mean:
\begin{align}
  Z_{ij} \sim \Normal\left(\frac{z_i + z_j}{2}, V_{ij}\right) \label{eq:inf}
\end{align} 
Here, \(V_{ij}\) is referred to as the \emph{segregation variance} in family
\((i,j)\).
This is the variation generated among offspring from the same parental pair due
to random Mendelian segregation in meiosis.
This approximation can be justified as arising from the limit where the number
of loci determining the trait tends to infinity \citep{barton2017}.

An equivalent, and for our purposes useful, way to characterize the model in
diploids and polyploids is to write \(Z_{ij} = Y_i + Y_j\), where \(Y_i\) and
\(Y_j\) are independent Gaussian random variables \(Y_i \sim
\Normal\left(\frac{z_i}{2}, V_i\right)\) (and similarly for \(Y_j\)).
We refer to \(Y_i\) as the (random) gametic value of individual \(i\), and to
\(V_i\) as the \emph{gametic segregation variance} of individual \(i\)
(clearly $V_{ij} = V_i + V_j$).
This formulation is helpful in that it highlights that Mendelian segregation
occurs independently in both parents when gametes are produced, which then
combine additively to determine the offspring trait value.

If we assume a population consisting of infinitely many unrelated
individuals, expressing a trait with genetic variance \(V_z\) and
segregation variance \(V_0\), we find that under random mating the
variance in the offspring generation is
\begin{align}
V_z' &= \Ex[\Var[Z_{ij}|Z_i, Z_j]] + \Var[\Ex[Z_{ij}|Z_i,Z_j]] \nonumber \\
   &= \Ex[V_{ij}] + \Var\left(\frac{Z_i + Z_j}{2}\right) \nonumber \\
   &= V_0 + \frac{V_z}{2}\ .
   \label{eq:eq}
\end{align}
So that at equilibrium (\(V_z'=V_z\)), we have \(V_z = 2V_0\) \citep{barton2017}.

Denoting by \(X\) the contribution to the trait value associated with a
randomly sampled haploid genome from the population, we also have that, at
equilibrium, \(V_z=2V_0=mV\), where \(V = \Var[X]\) and $m$ is the ploidy
level.
Note that no assumptions regarding $m$ have been made in the definition of the
model.
Indeed, as long as all individuals are of the same \emph{even} ploidy level,
\cref{eq:inf} and \cref{eq:eq} apply equally well to the inheritance of an
additive trait in polyploids as in diploids.
Two factors however do lead to differences with the standard diploid model due
to their effect on \(V_i\):
(1) the potential occurrence of double reduction in polyploids affects the
relationship between \(V\) and the segregation variance and 
(2) the evolution of the segregation variance over time in finite populations
is affected by the ploidy level.


\textbf{Perhaps we should organize the text around the table below. First
outline mixed-ploidy population, then indicate that we need the variances
associated with the different gametes that are produced. Gamete formation
occurs independently, so that is all we need. So start with the infinitesimal
model framed in terms of uniting gametes, and highlight that this works for any
ploidy level, and mixed-ploidy populations as long as we can work out the
segregation variances.}

\begin{table}[]
\begin{tabular}{l|rr}
cytotype   & haploid gamete variance & diploid gamete variance             \\ \cline{1-3}
diploid    & $\frac{1}{2}(1- F)V$    & $2\alpha_2(1-F)V$  \\
triploid   & $\frac{2}{3}(1-F)V$     & $\frac{2}{3}(1 + \alpha_3)(1-F)V$   \\
tetraploid & --                      & $(1+2\alpha_4)(1-F)V$             
\end{tabular}%
\end{table}

\section*{Results}

\section*{Discussion}

\section*{Competing interests}
No competing interest is declared.

\section*{Author contributions statement}
Not appliccable.

\section*{Data availability statement}

The authors affirm that all data necessary for confirming the conclusions of
the article are present within the article, figures, and supplementary
material. Software implementing the numerical methods and individual-based
simulations is available at \texttt{https://github.com/arzwa/InfGenetics}.

\section*{Acknowledgments}

This work was funded by the European Union (ERC BryoFit 101041201 granted to
CF). Views and opinions expressed are however those of the author(s) only and
do not necessarily reflect those of the European Union or the European Research
Council. Neither the European Union nor the granting authority can be held
responsible for them.

%%%%%%%%%%%%%%

\begin{appendices}

\end{appendices}

%\bibliographystyle{plain}
    %\bibliography{/home/arthur_z/vimwiki/bib.bib}


%USE THE BELOW OPTIONS IN CASE YOU NEED AUTHOR YEAR FORMAT.
\bibliographystyle{abbrvnat}
\bibliography{/home/arthur_z/vimwiki/bib.bib}

\clearpage
\setcounter{page}{1}


%SUPPLEMENT
\clearpage
\renewcommand{\thefigure}{S\arabic{figure}}
\renewcommand{\thesection}{S\arabic{section}}
\renewcommand{\thealgorithm}{S\arabic{algorithm}}
\setcounter{figure}{0}
\setcounter{section}{0}
\setcounter{equation}{0}
\setcounter{algorithm}{0}

\onecolumn % for one column layouts

\section{Supplementary figures}

\section{Supplementary Information}

\subsection*{Double reduction in autotetraploids}

When an autopolyploid forms multivalents during prophase I, a form of `internal
inbreeding' may occur as a result of the phenomenon called \emph{double
reduction} \citep{huang2019}.
Double reduction happens when, as a result of recombination, replicated gene
copies on sister chromatids move to the same pole during anaphase I (see
\cite{stift2010} or \cite{huang2019} for helpful illustrations).
The frequency of double reduction at a locus in the presence of multivalent
formation is determined by the frequency at which that locus is involved in a
cross-over (which depends on the distance to the centromere), and, in
tetraploids, has an upper bound at \(1/6\) \citep{mather1935,stift2010}.
 
Consider a locus with genotype $ABCD$ in an autotetraploid.
In the absence of double reduction, six distinct gametes are produced with
equal frequencies.
When double reduction happens, four additional types of gametes carrying
replicated gene copies ($AA, BB, CC$ or $DD$ gametes) will be produced with
equal frequencies.
The segregation variance will hence be increased in the presence of double
reduction, as more distinct types of gametes are produced compared to disomic
inheritance.
For a random genotype \(X_1X_2X_3X_4\), we can find the gametic segregation
variance contributed by a locus when double reduction happens as
\begin{align*}
    \Ex[\var[Y&|X_1,X_2,X_3,X_4]] = \var[Y] - \Var[\Ex[Y|X_1,X_2,X_3,X_4]] \\
        &= \var[2X] - \var\left[\frac1 4 (2X_1 + 2X_2 + 2X_3 + 2X_4)\right] \\
        &= 4v - \frac{1}{4}4v
        = 3v
\end{align*} where \(X\) denotes the additive effect of a random allele
at the locus drawn from the reference population and \(v = \Var[X]\).
In the absence of double reduction we have
\begin{align*}
    \Ex[\var[Y&|X_1,X_2,X_3,X_4]] \\
        &= 2\Var[X] - \var\left[\frac1 6 \sum_{i=1}^3 \sum_{j=i+1}^4(X_i+X_j)\right]
        = v
\end{align*}
Assuming that the probability of double reduction at any locus is \(\alpha\)
(also \(\alpha_4\) below), and summing over independent loci, we find that the
gametic segregation variance in the presence of double reduction should be
\begin{equation}
\frac{V_0}{2} = (1-\alpha)V + 3\alpha V = V(1+2\alpha).
\label{eq:dr}
\end{equation}

\end{document}
